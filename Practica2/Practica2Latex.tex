\documentclass[a4paper,10pt]{article}
\usepackage[utf8]{inputenc}

%opening
\title{Práctica 2 - \LaTeX } % Cambia el titulo por alguno que tu elijas.
\author{Luis Felipe Samaniego Bernal} % Cambialo por tu nombre completo.

\begin{document}

\maketitle

% \begin{abstract}
% 
% \end{abstract}

\section{Seccion 1}


Ahora se me ocurrió: ... la independencia de la aceleración gravitacional de la naturaleza de la caída se comportan como lo hacen en un espacio libre de gravitación. 


Esta sustancia, puede expresarse de la siguiente manera: En un campo gravitacional (de pequeña extensión espacial) las cosas sucedió en 1908. ¿Por qué se necesitaron otros siete años para la construcción de la teoría general de la relatividad?


La principal razón radica en que no es tan fácil liberarse de la idea de que las coordenadas deben tener un significado métrico inmediato.
\begin{flushright}
ALBERT EINSTEIN [en Schilpp (1949), págs. 65-67]
\end{flushright}

\[ G_{\mu\nu} = 8 \pi T_{\mu\nu} \]



\end{document}
